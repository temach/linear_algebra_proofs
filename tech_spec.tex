\documentclass[a4paper,12pt,twoside]{article}

% Packages required by doxygen
\usepackage[export]{adjustbox} % also loads graphicx
\usepackage{graphicx}
\usepackage[utf8]{inputenc}
\usepackage{multicol}
\usepackage{multirow}
\usepackage{makeidx}
\usepackage[table]{xcolor}
\usepackage{hyperref} % for math mode in section titles
\usepackage{fixltx2e} % for math mode in section titles
\usepackage{amsmath, amsthm, amssymb}    % for maths formulae nice display

% NLS support packages
\usepackage[T2A]{fontenc}
\usepackage[russian]{babel}

% Font selection
\usepackage[T1]{fontenc}
\usepackage[scaled=.90]{helvet}
\usepackage{courier}
\usepackage{amssymb}
\usepackage{amsthm}

% Page & text layout
\usepackage{geometry}
\geometry{%
  a4paper,%
  top=2.5cm,%
  bottom=2.5cm,%
  left=2.5cm,%
  right=2.5cm%
}
%\setlength{\emergencystretch}{15pt}
\setlength{\parindent}{0cm}
\setlength{\parskip}{0.2cm}

% Headers & footers
\usepackage{fancyhdr}
\pagestyle{fancyplain}
\fancyhead[LE]{\fancyplain{}{\bfseries\thepage}}
\fancyhead[CE]{\fancyplain{}{\scriptsize\textbf{Ну понять, так оно понятно}}}
\fancyhead[RE]{\fancyplain{}{}}
\fancyhead[LO]{\fancyplain{}{}}
\fancyhead[CO]{\fancyplain{}{\scriptsize\textbf{Ну понять-то оно понятно}}}
\fancyhead[RO]{\fancyplain{}{\bfseries\thepage}}
\fancyfoot[LE]{\fancyplain{}{}}
\fancyfoot[CE]{\fancyplain{}{}}
\fancyfoot[RE]{\fancyplain{}{}}
\fancyfoot[LO]{\fancyplain{}{}}
\fancyfoot[CO]{\fancyplain{}{}}
\fancyfoot[RO]{\fancyplain{}{}}

% Indices & bibliography
\usepackage{natbib}
\usepackage[titles]{tocloft}
\setcounter{tocdepth}{3}
\setcounter{secnumdepth}{5}
\makeindex

% change style of titles in \section{}
\usepackage{titlesec}
\titleformat{\section}[hang]{\Large\bfseries\center}{\thetitle.}{1em}{}
\titleformat{\subsection}[hang]{\large\normalfont\raggedright}{\thetitle.}{1em}{\underline}
\titleformat{\subsubsection}[hang]{\normalsize\normalfont\raggedright}{\thetitle.}{1pt}{}

% Packages for text layout in normal mode
% \usepackage[parfill]{parskip} % автоматом делает пустые линии между параграфами, там где они есть в тексте
% \usepackage{indentfirst} % indent even in first paragraph
\usepackage{setspace}	 % controls space between lines
\setstretch{1} % space between lines
\setlength\parindent{0.9cm} % size of indent for every paragraph
\usepackage{csquotes}% превратить " " в красивые двойные кавычки
\MakeOuterQuote{"}

\renewcommand{\qedsymbol}{$\blacksquare$ - доказано нормально.}

% this makes items spacing single-spaced in enumerations.
\newenvironment{my_enumerate}{
\begin{enumerate}
  \setlength{\itemsep}{1pt}
  \setlength{\parskip}{0pt}
  \setlength{\parsep}{0pt}}{\end{enumerate}
}


% Custom commands
\DeclareMathAlphabet{\mathpzc}{OT1}{pzc}{m}{it}

% Custom packages
\usepackage{pdfpages}

\makeindex


\usepackage{pgfplots} % for graph of parabola in 45
% arrows as stealth fighters
\tikzset{>=stealth}


%===== C O N T E N T S =====

\begin{document}

% Titlepage & ToC
\pagenumbering{roman}

\begin{titlepage}
\begin{center}
\vspace*{0.7cm}
{\large ПРАВИТЕЛЬСТВО РОССИЙСКОЙ ФЕДЕРАЦИИ \\
НАЦИОНАЛЬНЫЙ ИССЛЕДОВАТЕЛЬСКИЙ УНИВЕРСИТЕТ \\
«ВЫСШАЯ ШКОЛА ЭКОНОМИКИ» }\\
\vspace*{0.2cm}
Факультет компьютерных наук \\
\smallskip
{\small Департамент программнoй инженерии \\
}
\vspace{1cm}

{\textbf{Линейная алгебра}} \\
\medskip

Москва \number\year \\
\end{center}
\end{titlepage}

\newpage

\newpage
\tableofcontents
\pagenumbering{arabic}

\newpage

% --- add my custom headers ---
\section{1 Сформулировать и доказать теорему о методе Гаусса}
Любую матрицу можно привести к ступенчатому виду с помощью элементарных преобразований.
В доказательство предъявим алгоритм, который гарантированно завершится для конечных матриц:
\begin{my_enumerate}
\item[Шаг 1]
\begin{my_enumerate}
\item Фиксируем элемент в верхнем левом углу $ a_{ij} $
\item Если $a_{ij} = 0$ : переходим к шагу 2. \\
Иначе объявляем этот элемент ведущим.
\item Используя ведущий элемент будем добиваемся того, что элементы во всех строках ниже него обнуляются. Если $ a_{ij} $ - ведущий, то сложим строки $ k, ( k \neq i ) $ и $i$. Причем строку $i$ домножим на $ \lambda = - \frac{a_{kj}}{a_{ij}}$.
\item Смещаемся в матрице на строку вниз и на столбец вправо. Переходим к шагу 1. (пока не закончится матрица)
\end{my_enumerate}

\item[Шаг 2] Если текущий элемент нулевой, просматриваем все элементы под ним: Если найдем ненулевой в строке $k$ то меняем строки $k$ и $i$ местами и переходим к шагу 1. Если все нулевые переходим к шагу 3.

\item[Шаг 3] Переходим на один столбец вправо. Если справа есть столбцы переходим к шагу 1. Если столбцы закончились - конец алгоритма. 

\end{my_enumerate}

\section{43 Сформулировать и доказать теорему об инвариантности ранга матрицы квадратичной формы. }
Ранг матрицы квадратичной формы не меняется в зависимости от базиса.
\[
Rg( A_{f} ) = Rg( S_{e \rightarrow f}^{{\small T}} A_{e} S_{e \rightarrow f} )
\]
Матрица S это преобразование между двумя 
ортонормиованными системами координат
$ \implies $ S ортагональна  $ \implies $
$ \implies \exists S^{-1} = S^{{\small T}} $
$ \implies det S  \neq 0 $
S - невырожденная матрица.

Домножение матрицы $А$ на невырожденную матрицу $B$ не меняет ранг матрицы $A$. Это отдельная теорема, докажем ее. 
Очевидно что $ Rg( A B ) \leq Rg(A) $, а также $ Rg( A B B^{-1} ) \leq Rg( A B ) $, следовательно
$ Rg(A) \leq Rg(A B) \leq Rg( A ) = Rg(A) $

Поэтому:
$ Rg( A_{f} ) = Rg( S_{e \rightarrow f}^{{\small T}} A_{e} S_{e \rightarrow f} ) $

\section{44 Сформулируйте и докажите теорему о приведении квадратичных форм к диагональному (каноническому) виду при помощи ортагональной замены координат. }
\begin{enumerate}
\item Рассмотрим квадратичную форму $q(x)$
\item Возьмем стандартный ортонормированный базис $e = {e_1, \dots e_n }$.
\item Составим матрицу этой квадратичной формы $A_e$. 
Она будет симметрична, $A^{{\small T}} = A  \implies A $ - самосопряженная матрица.
% как доказать что самосопряженный оператор гарантированно диагонализируем?
% смотри страницу 132 кострикин 2 том

\item Докажем что самосопряженный оператор всегда диагонализируем.
Докажем две леммы, а затем приведем основное доказательство.
\emph{Лемма. Собственные значения самосопряженного оператора будут вещественными.}

\begin{proof} 
Обозначим через $ (a, b) $ скалярное произведение векторов $a$ и $b$. 
Пусть $A : V \rightarrow V $ – самосопряженный оператор (то есть $ A = A^{*}$), 
$ \lambda $ - его собственное значение, 
$ e $ - вектор отвечающий этому собственному значению.

По определению 
$ \lambda (e,e) = (\lambda e,e) = (Ae,e) = (e,A^{*} e) = (e,Ae) = (e,\lambda e) = \bar{\lambda} (e,e)$
Здесь $ \bar{\lambda} $ это комплексное сопряжение $ \lambda $. 

Так как $ (e,e) \neq 0 \implies \bar{\lambda} = \lambda $.
\end{proof}

\emph{Лемма. У каждого симметричного линейного оператора существует собственный вектор.}
\begin{proof}
Как и любой вещественный оператор, самосопряженный оператор 
обладает 1 или 2-х мерным собственным подпространством.
Существование одномерного инвариантного пространства совпадает с утверждением леммы.
Расмотрим случай когда 


\end{proof}


\item У каждого самосопряженного оператора есть 1 или 2 инвариантных подпространства.
\item Если одно под-пространство. Есть решение.
\item Если два подпространства. Тоже обязательно будут решения.
\item Рассмотрим ее как матрицу оператора $\phi$. Об операторах уже много чего известно, воспользуемся этим.
\item Так как матрица $\phi_e$ симметрична $ \implies $ оператор диагонализируем. (Именно поэтому любую кв. форму можно привести к канон виду). Будем приводить оператор $\phi$ к диагональному виду.  
\item Найдем собственные значения $\lambda_1, \dots, \lambda_n $.
\item Найдем собственные вектора.
\item Процессом ортагонализации Грамма-Шмидта добьемся того 
чтобы все собственные вектора были ортагональны. Далее нормируем каждый вектор. 
В результате получим набор векторов $\{ f_1, \dots, f_n \}$
, которые образуют ортонормированный базис $f$.
\item Для операторов верно: {\large $\phi_e = T_{e \rightarrow f} \phi_f T_{e \rightarrow f}^{-1}$ }
\begin{enumerate}
\item[] $ T_{e \rightarrow f} $ - матрица оператора перехода от базиса $e$ к базису $f$ (матрица из векторов базиса $f$ записанных по столбцам, она будет ортагональна).
\item[] $ \phi_{f} $ - матрица оператора в базисе $f$ (будет диагональной матрицей).
\item[] $ \phi_{e} $ - матрица оператора в изначальном базисе $e$ (симметричная матрица).
\end{enumerate}
\item Базисы $e$ и $f$ ортонормированны 
$\iff $ 
оператор перехода $T_{e \rightarrow f} $ будет ортагональным
$ \implies $
$ T_{e \rightarrow f}^{-1} = T_{e \rightarrow f}^{{\small T}}  $
\[
\phi_e = T_{e \rightarrow f} \phi_f T_{e \rightarrow f}^{-1}
\]
\[
\phi_{f} = T_{e \rightarrow f}^{-1} \phi_{e} T_{e \rightarrow f}
\]
\[
\phi_{f} = T_{e \rightarrow f}^{{\small T}} \phi_{e} T_{e \rightarrow f}
\]
\item Вернемся к квадратичным формам и перепишем матрицу оператора как кв. форму:
\[
A_{f} = T_{e \rightarrow f}^{{\small T}} A_{e} T_{e \rightarrow f}
\]
Где $ A_{f} $ - построенная диагональная матрица квадратичной формы. 
$ T_{e \rightarrow f} $ - ортагональная матрица перехода (применение такой матрицы это ортагональная замена координат). 
$ A_{e} $ - изначальная матрица кв. формы. \\
(или по формуле стас: $A_{e'} = S_{e \rightarrow e'}^{{\small T}} A_{e} S_{e \rightarrow e'} $ )
\item Теперь по матрице $A_{f}$ мы можем выписать канонический вид кв. формы $q(x)$, который мы получили при помощи ортагональной замены координат.
\item Так как все манипуляции логичны и не запрещенны $ \implies $ $ \forall q(x) $можно привести к канон. виду с помощью оратагональной замены координат.
\end{enumerate}


\section{45 Дайте определение параболы как геометрического места точек. Выведите  ее каноническое уравнение.}
(Мы уже знаем что каноническое уравнение параболы это $y^2 = 2px$, осталось это показать.)  \\
Геометрические точки задающие параболу равноудалены от фокуса пораболы и 
директрисы (смотри википедию "эксцентриситет").
В каноническом уравнении параболы (картинка ниже также дана для канонического случая):
\begin{enumerate}
\item[] Фокус  – точка $ F(p/2, 0) $. 
\item[] Директриса – вертикальная прямая $L$ заданная 
уравнением: $ x = -p/2 $.
\end{enumerate}

\begin{center}
\begin{tikzpicture}
    \begin{axis}[
            axis lines=middle,
            axis line style=->,
            xmin=-8,xmax=8,
            ymin=-5,ymax=5,
            xlabel=$x$,
            ylabel=$y$,
            xtick=\empty,
            ytick=\empty,
            xticklabels=\empty,
            yticklabels=\empty,
        ]
        \addplot[smooth,thick, blue,-]({x^2},{x});
        \addplot[smooth,thick, red,-]({-2},{x})
        [yshift=105pt, xshift=-25pt]            
            node[pos=0] {$L \left( -\frac{p}{2}, 0 \right)$}
        ;      
        \addplot+[only marks, red] coordinates{(2,0) (4,2)}
        [yshift=10pt, xshift=12pt]            
            node[pos=0] {$F \left(\frac{p}{2}, 0 \right)$}
            node[pos=1] {$ M(x,y) $}
        ;
    \end{axis}
\end{tikzpicture}
\end{center}

\smallskip 

Пусть $\rho(A,B) $ – дистанция между точками $A$ и $B$. Рассмотрим точку $M(x,y)$. 
Точка $M\in$ параболе $\iff$ выполнятся соотношение
$\rho(F,M) = \rho(F,L)$. Перепишем уравнение относительно координат:
\[
\sqrt{ \left( x - \frac{p}{2} \right)^2 + y^2  } = | x + \frac{p}{2} |
\]
Возведем обе стороны в квадрат:
\[
\left( x - \frac{p}{2} \right)^2 + y^2 =  \left( x + \frac{p}{2} \right)^2
\]
Раскрываем скобки и переносим:
\[
y^2 = 2px
\]
Каноническое уравнение. Здесь $2p$ просто коэффициент. 
(Каноническое уравнение параболы записывают через $y^2 = 2px$ 
вместо $y^2 = kx$, так как у параметра $p$ появляется геометрическое 
значение, это параметр параболы)






\section{46 Проведите полное исследование уравнения \texorpdfstring{$ Ax^2 + Cy^2 + Dx + Ey + F = 0, A^2 + C^2 > 0$}{TEXT} кривой второго порядка}

Сначала перепишем уравнение в более хорошей форме:
\[
    A \left(  x^2 + 2\frac{D}{2A}x + \frac{D^2}{4A^2}  \right) - \frac{D^2}{4A} 
+ C \left(  y^2 + 2\frac{E}{2C}y + \frac{E^2}{4C^2}  \right) - \frac{E^2}{4C}
+ F = 0 
\]

\[
    A \left( x + \frac{D}{2A} \right) ^2 
+ C \left( y + \frac{E}{2C}  \right) ^2 
+ \left( F - \frac{E^2}{4C}  - \frac{D^2}{4A} \right) = 0
\]

Для параллельного переноса координат, сделаем замену 
$x' = x + \frac{D}{2A}$, а также 
$y' = y + \frac{E}{2C}$. 
Заодно заменим константу:
$F' = F - \frac{E^2}{4C}  - \frac{D^2}{4A}$
\[
A(x')^2 + C(y')^2 + F' = 0
\]
\[
A(x')^2 + C(y')^2 = - F'
\]

Теперь рассмотрим 3 варианта (будем строит дерево вариантов):
\begin{enumerate}
%============================
\item[Вариант 1] $AC > 0$
\begin{enumerate}
\item  $ F' = 0 \implies Ax'^2 + Cy'^2 = 0 \iff x'^2 = 0, y'^2 = 0$. То есть решение это точка на 
плоскости $x = - \frac{D}{2A}, y = - \frac{E}{2C}$.

\item $ F' \neq 0 \implies \frac{x'^2}{-F' / A} +  \frac{y'^2}{-F' / C} = 1$ (Разделили на $ - F'$ и перенесли коэффициенты $A, C$ вниз). 
\begin{enumerate}
\item $ F' < 0 \implies  (-F' / A) > 0 , (-F' / C)  > 0 $. 
Решение можо переписать в виде $ \frac{x^2}{a^2} + \frac{y^2}{b^2} = 1 $. Это эллипс.
\item $ F' > 0 \implies -F' < 0 $. Так как рассматриваем 
$ AC > 0 $ следовательно решение для $ A(x')^2 + C(y')^2 = - F' $ это пустое множество. 
\end{enumerate}

\end{enumerate}

\bigskip
%============================
\item[Вариант 2] $AC < 0$
\begin{enumerate}
\item  $ F' = 0 \implies Ax'^2 + Cy'^2 = 0 $.  \\
Разложим по $ ( a^2 - b^2 ) = (a - b)(a + b) $, так как либо $A < 0$,  либо $ C < 0 $.
Получим либо $ (\sqrt{|A|}x' - \sqrt{|C|}y')(\sqrt{|A|}x' + \sqrt{|C|}y') = 0 $, \\
либо $ (- \sqrt{|A|}x' + \sqrt{|C|}y')(\sqrt{|A|}x' + \sqrt{|C|}y') = 0 $. \\
В любом случае верно что:
$ \sqrt{|A|}x' = \sqrt{|C|}y', \, \sqrt{|A|}x' = - \sqrt{|C|}y' $.
 Это пара прямых на плоскости.

\item $ F' \neq 0 \implies \frac{x'^2}{-F' / A} +  \frac{y'^2}{-F' / C} = 1$ 
\begin{enumerate}
\item $  F' > 0 и (A < 0, C > 0) \implies  (-F' / A) > 0,  (-F' / C)  < 0 $. 
Решение можо переписать в виде $ \frac{x^2}{a^2} - \frac{y^2}{b^2} = 1 $. 
Это гипербола.
\item $ F' < 0 и (A > 0, C < 0) \implies   (-F' / A) < 0,  (-F' / C)  > 0 $. 
Решение можно переписать в виде $ - \frac{x^2}{a^2} + \frac{y^2}{b^2} = 1 $. 
Это сопряженная гипербола.
\end{enumerate}

\end{enumerate}

\bigskip
%============================
\item[Вариант 3] $AC = 0 $

\begin{enumerate}

\item $ A \neq 0, C = 0 \implies Ax^2 + Dx + Ey + F = 0 $ \\
Выделим полный квадрат по $x$:
\[
A \left( x + \frac{D}{2A} \right)^2 + Ey + F - \frac{D^2}{4A} = 0
\]
Введем 
$ F' = F - \frac{D}{4A} \implies A \left( x + \frac{D}{2A} \right)^2 = -Ey - F' $

\begin{enumerate}
\item Если $ E =  0 \implies A \left( x + \frac{D}{2A} \right)^2 = -F' $
\begin{enumerate}
\item $F' = 0 \implies \left( x + \frac{D}{2A} \right)^2 = 0 $. 
Это вертикальная прямая $ x = \frac{-D}{2A} $
\item $ F' \neq 0 $ Перепишем уравнение в виде 
$ \left( x + \frac{D}{2A} \right)^2 = \frac{-F'}{A} $ \\
При $ signF' = signA \implies \frac{-F'}{A} < 0 \implies \emptyset $.  \\
При $ signF' = - signA  \implies x = \pm \sqrt{\frac{-F'}{A}} - \frac{D}{2A} $. Это пара вертикальных прямых.
\end{enumerate}

\item Если $ E \neq 0 \implies A \left( x + \frac{D}{2A} \right)^2 = -E( y + \frac{F'}{E} ) $.
Осуществим параллельный перенос системы координат 
$ x' = x + \frac{D}{2A}, y' = y + \frac{F'}{E} $. Получим:
\[
{x'}^2 = - \frac{E}{A} y'  \iff y' = - \frac{A}{E} {x'}^2
\]
Это парабола (обыкновенная, вдоль оси $y$)
\end{enumerate}

\item $ A = 0, C \neq 0 \implies Cy^2 + Dx + Ey + F = 0  $ \\
Разбор случая полностью аналогичен предыдущему. Вертикальные линии станут горизонтальными. Парабола будет вдоль оси х.
\end{enumerate}
\end{enumerate}





% Index
\newpage
% \phantomsection
% \addcontentsline{toc}{section}{Алфавитный указатель}    
\printindex

\end{document}
