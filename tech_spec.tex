\documentclass[a4paper,12pt]{article}

% Packages required by doxygen
\usepackage[export]{adjustbox} % also loads graphicx
\usepackage{graphicx}
\usepackage[utf8]{inputenc}
\usepackage{multicol}
\usepackage{multirow}
\usepackage{makeidx}
\usepackage[table]{xcolor}
\usepackage{hyperref} % for math mode in section titles
\usepackage{fixltx2e} % for math mode in section titles
\usepackage{amsmath, amsthm, amssymb}    % for maths formulae nice display

% NLS support packages
\usepackage[T2A]{fontenc}
\usepackage[russian]{babel}

% Font selection
\usepackage[T1]{fontenc}
\usepackage[scaled=.90]{helvet}
\usepackage{courier}
\usepackage{amssymb}

% Page & text layout
\usepackage{geometry}
\geometry{%
  a4paper,%
  top=2.5cm,%
  bottom=2.5cm,%
  left=2.5cm,%
  right=2.5cm%
}
%\setlength{\emergencystretch}{15pt}
\setlength{\parindent}{0cm}
\setlength{\parskip}{0.2cm}

% Headers & footers
\usepackage{fancyhdr}
\pagestyle{fancyplain}
\fancyhead[LE]{\fancyplain{}{\bfseries\thepage}}
\fancyhead[CE]{\fancyplain{}{\scriptsize\textbf{RU.17701729.503200 ТЗ 01-1-ЛУ}}}
\fancyhead[RE]{\fancyplain{}{}}
\fancyhead[LO]{\fancyplain{}{}}
\fancyhead[CO]{\fancyplain{}{\scriptsize\textbf{RU.17701729.503200 ТЗ 01-1-ЛУ}}}
\fancyhead[RO]{\fancyplain{}{\bfseries\thepage}}
\fancyfoot[LE]{\fancyplain{}{}}
\fancyfoot[CE]{\fancyplain{}{}}
\fancyfoot[RE]{\fancyplain{}{}}
\fancyfoot[LO]{\fancyplain{}{}}
\fancyfoot[CO]{\fancyplain{}{}}
\fancyfoot[RO]{\fancyplain{}{}}

% Indices & bibliography
\usepackage{natbib}
\usepackage[titles]{tocloft}
\setcounter{tocdepth}{3}
\setcounter{secnumdepth}{5}
\makeindex

% change style of titles in \section{}
\usepackage{titlesec}
\titleformat{\section}[hang]{\Large\bfseries\center}{\thetitle.}{1em}{}
\titleformat{\subsection}[hang]{\large\normalfont\raggedright}{\thetitle.}{1em}{\underline}
\titleformat{\subsubsection}[hang]{\normalsize\normalfont\raggedright}{\thetitle.}{1pt}{}

% Packages for text layout in normal mode
% \usepackage[parfill]{parskip} % автоматом делает пустые линии между параграфами, там где они есть в тексте
% \usepackage{indentfirst} % indent even in first paragraph
\usepackage{setspace}	 % controls space between lines
\setstretch{1} % space between lines
\setlength\parindent{0.9cm} % size of indent for every paragraph
\usepackage{csquotes}% превратить " " в красивые двойные кавычки
\MakeOuterQuote{"}


% this makes items spacing single-spaced in enumerations.
\newenvironment{my_enumerate}{
\begin{enumerate}
  \setlength{\itemsep}{1pt}
  \setlength{\parskip}{0pt}
  \setlength{\parsep}{0pt}}{\end{enumerate}
}


% Custom commands


% Custom packages
\usepackage{pdfpages}

\makeindex

%===== C O N T E N T S =====

\begin{document}

% Titlepage & ToC
\pagenumbering{roman}

\begin{titlepage}
\begin{center}
\vspace*{0.7cm}
{\large ПРАВИТЕЛЬСТВО РОССИЙСКОЙ ФЕДЕРАЦИИ \\
НАЦИОНАЛЬНЫЙ ИССЛЕДОВАТЕЛЬСКИЙ УНИВЕРСИТЕТ \\
«ВЫСШАЯ ШКОЛА ЭКОНОМИКИ» }\\
\vspace*{0.2cm}
Факультет компьютерных наук \\
\smallskip
{\small Департамент программнoй инженерии \\
}
\vspace{1cm}

{\textbf{Спасалка по линейной алгебре}} \\
\medskip

Москва \number\year \\
\end{center}
\end{titlepage}

\newpage

{\large Ну понять–то оно понятно}

\newpage
\tableofcontents
\pagenumbering{arabic}

\newpage

% --- add my custom headers ---
\section{1 Сформулировать и доказать теорему о методе Гаусса}
Любую матрицу можно привести к ступенчатому виду с помощью элементарных преобразований.
В доказательство предъявим алгоритм, который гарантированно завершится для конечных матриц:
\begin{my_enumerate}
\item[Шаг 1]
\begin{my_enumerate}
\item Фиксируем элемент в верхнем левом углу $ a_{ij} $
\item Если $a_{ij} = 0$ : переходим к шагу 2. \\
Иначе объявляем этот элемент ведущим.
\item Используя ведущий элемент будем добиваемся того, что элементы во всех строках ниже него обнуляются. Если $ a_{ij} $ - ведущий, то сложим строки $ k, ( k \neq i ) $ и $i$. Причем строку $i$ домножим на $ \lambda = - \frac{a_{kj}}{a_{ij}}$.
\item Смещаемся в матрице на строку вниз и на столбец вправо. Переходим к шагу 1. (пока не закончится матрица)
\end{my_enumerate}

\item[Шаг 2] Если текущий элемент нулевой, просматриваем все элементы под ним: Если найдем ненулевой в строке $k$ то меняем строки $k$ и $i$ местами и переходим к шагу 1. Если все нулевые переходим к шагу 3.

\item[Шаг 3] Переходим на один столбец вправо. Если справа есть столбцы переходим к шагу 1. Если столбцы закончились - конец алгоритма. 

\end{my_enumerate}


\section{46 Проведите полное исследование уравнения \texorpdfstring{$ Ax^2 + Cy^2 + Dx + Ey + F = 0, A^2 + C^2 > 0$}{TEXT} кривой второго порядка}

Сначала перепишем уравнение в более хорошей форме:
\[
    A \left(  x^2 + 2\frac{D}{2A}x + \frac{D^2}{4A^2}  \right) - \frac{D^2}{4A} 
+ C \left(  y^2 + 2\frac{E}{2C}y + \frac{E^2}{4C^2}  \right) - \frac{E^2}{4C}
+ F = 0 
\]

\[
    A \left( x + \frac{D}{2A} \right) ^2 
+ C \left( y + \frac{E}{2C}  \right) ^2 
+ \left( F - \frac{E^2}{4C}  - \frac{D^2}{4A} \right) = 0
\]

Для параллельного переноса координат, сделаем замену 
$x' = x + \frac{D}{2A}$, а также 
$y' = y + \frac{E}{2C}$. 
Заодно заменим константу:
$F' = F - \frac{E^2}{4C}  - \frac{D^2}{4A}$
\[
A(x')^2 + C(y')^2 + F' = 0
\]
\[
A(x')^2 + C(y')^2 = - F'
\]

Теперь рассмотрим 3 варианта:
\begin{my_enumerate}
%============================
\item[Вариант 1] $AC > 0$
\begin{my_enumerate}
\item[$F' = 0$]  $ \implies Ax'^2 + Cy'^2 = 0 \iff x'^2 = 0, y'^2 = 0$. То есть решение это точка на плоскости $x = - \frac{D}{2A}, y = - \frac{E}{2C}$.
\item[$F' \neq 0$] $ \implies \frac{x'^2}{-F' / A} +  \frac{y'^2}{-F' / C} = 1$ (Разделили на $ - F'$ и перенесли коэффициенты $A, C$ вниз).




\end{my_enumerate}

%============================
\item[AC < 0]

%============================
\item[AC = 0]

\end{my_enumerate}

% Index
\newpage
% \phantomsection
% \addcontentsline{toc}{section}{Алфавитный указатель}    
\printindex

\end{document}
